% Options for packages loaded elsewhere
\PassOptionsToPackage{unicode}{hyperref}
\PassOptionsToPackage{hyphens}{url}
\PassOptionsToPackage{dvipsnames,svgnames,x11names}{xcolor}
%
\documentclass[
  10,
  twocolumn]{article}

\usepackage{amsmath,amssymb}
\usepackage{iftex}
\ifPDFTeX
  \usepackage[T1]{fontenc}
  \usepackage[utf8]{inputenc}
  \usepackage{textcomp} % provide euro and other symbols
\else % if luatex or xetex
  \usepackage{unicode-math}
  \defaultfontfeatures{Scale=MatchLowercase}
  \defaultfontfeatures[\rmfamily]{Ligatures=TeX,Scale=1}
\fi
\usepackage{lmodern}
\ifPDFTeX\else  
    % xetex/luatex font selection
\fi
% Use upquote if available, for straight quotes in verbatim environments
\IfFileExists{upquote.sty}{\usepackage{upquote}}{}
\IfFileExists{microtype.sty}{% use microtype if available
  \usepackage[]{microtype}
  \UseMicrotypeSet[protrusion]{basicmath} % disable protrusion for tt fonts
}{}
\makeatletter
\@ifundefined{KOMAClassName}{% if non-KOMA class
  \IfFileExists{parskip.sty}{%
    \usepackage{parskip}
  }{% else
    \setlength{\parindent}{0pt}
    \setlength{\parskip}{6pt plus 2pt minus 1pt}}
}{% if KOMA class
  \KOMAoptions{parskip=half}}
\makeatother
\usepackage{xcolor}
\setlength{\emergencystretch}{3em} % prevent overfull lines
\setcounter{secnumdepth}{-\maxdimen} % remove section numbering
% Make \paragraph and \subparagraph free-standing
\makeatletter
\ifx\paragraph\undefined\else
  \let\oldparagraph\paragraph
  \renewcommand{\paragraph}{
    \@ifstar
      \xxxParagraphStar
      \xxxParagraphNoStar
  }
  \newcommand{\xxxParagraphStar}[1]{\oldparagraph*{#1}\mbox{}}
  \newcommand{\xxxParagraphNoStar}[1]{\oldparagraph{#1}\mbox{}}
\fi
\ifx\subparagraph\undefined\else
  \let\oldsubparagraph\subparagraph
  \renewcommand{\subparagraph}{
    \@ifstar
      \xxxSubParagraphStar
      \xxxSubParagraphNoStar
  }
  \newcommand{\xxxSubParagraphStar}[1]{\oldsubparagraph*{#1}\mbox{}}
  \newcommand{\xxxSubParagraphNoStar}[1]{\oldsubparagraph{#1}\mbox{}}
\fi
\makeatother


\providecommand{\tightlist}{%
  \setlength{\itemsep}{0pt}\setlength{\parskip}{0pt}}\usepackage{longtable,booktabs,array}
\usepackage{calc} % for calculating minipage widths
% Correct order of tables after \paragraph or \subparagraph
\usepackage{etoolbox}
\makeatletter
\patchcmd\longtable{\par}{\if@noskipsec\mbox{}\fi\par}{}{}
\makeatother
% Allow footnotes in longtable head/foot
\IfFileExists{footnotehyper.sty}{\usepackage{footnotehyper}}{\usepackage{footnote}}
\makesavenoteenv{longtable}
\usepackage{graphicx}
\makeatletter
\def\maxwidth{\ifdim\Gin@nat@width>\linewidth\linewidth\else\Gin@nat@width\fi}
\def\maxheight{\ifdim\Gin@nat@height>\textheight\textheight\else\Gin@nat@height\fi}
\makeatother
% Scale images if necessary, so that they will not overflow the page
% margins by default, and it is still possible to overwrite the defaults
% using explicit options in \includegraphics[width, height, ...]{}
\setkeys{Gin}{width=\maxwidth,height=\maxheight,keepaspectratio}
% Set default figure placement to htbp
\makeatletter
\def\fps@figure{htbp}
\makeatother
% definitions for citeproc citations
\NewDocumentCommand\citeproctext{}{}
\NewDocumentCommand\citeproc{mm}{%
  \begingroup\def\citeproctext{#2}\cite{#1}\endgroup}
\makeatletter
 % allow citations to break across lines
 \let\@cite@ofmt\@firstofone
 % avoid brackets around text for \cite:
 \def\@biblabel#1{}
 \def\@cite#1#2{{#1\if@tempswa , #2\fi}}
\makeatother
\newlength{\cslhangindent}
\setlength{\cslhangindent}{1.5em}
\newlength{\csllabelwidth}
\setlength{\csllabelwidth}{3em}
\newenvironment{CSLReferences}[2] % #1 hanging-indent, #2 entry-spacing
 {\begin{list}{}{%
  \setlength{\itemindent}{0pt}
  \setlength{\leftmargin}{0pt}
  \setlength{\parsep}{0pt}
  % turn on hanging indent if param 1 is 1
  \ifodd #1
   \setlength{\leftmargin}{\cslhangindent}
   \setlength{\itemindent}{-1\cslhangindent}
  \fi
  % set entry spacing
  \setlength{\itemsep}{#2\baselineskip}}}
 {\end{list}}
\usepackage{calc}
\newcommand{\CSLBlock}[1]{\hfill\break\parbox[t]{\linewidth}{\strut\ignorespaces#1\strut}}
\newcommand{\CSLLeftMargin}[1]{\parbox[t]{\csllabelwidth}{\strut#1\strut}}
\newcommand{\CSLRightInline}[1]{\parbox[t]{\linewidth - \csllabelwidth}{\strut#1\strut}}
\newcommand{\CSLIndent}[1]{\hspace{\cslhangindent}#1}

\makeatletter
\@ifpackageloaded{caption}{}{\usepackage{caption}}
\AtBeginDocument{%
\ifdefined\contentsname
  \renewcommand*\contentsname{Table of contents}
\else
  \newcommand\contentsname{Table of contents}
\fi
\ifdefined\listfigurename
  \renewcommand*\listfigurename{List of Figures}
\else
  \newcommand\listfigurename{List of Figures}
\fi
\ifdefined\listtablename
  \renewcommand*\listtablename{List of Tables}
\else
  \newcommand\listtablename{List of Tables}
\fi
\ifdefined\figurename
  \renewcommand*\figurename{Figure}
\else
  \newcommand\figurename{Figure}
\fi
\ifdefined\tablename
  \renewcommand*\tablename{Table}
\else
  \newcommand\tablename{Table}
\fi
}
\@ifpackageloaded{float}{}{\usepackage{float}}
\floatstyle{ruled}
\@ifundefined{c@chapter}{\newfloat{codelisting}{h}{lop}}{\newfloat{codelisting}{h}{lop}[chapter]}
\floatname{codelisting}{Listing}
\newcommand*\listoflistings{\listof{codelisting}{List of Listings}}
\makeatother
\makeatletter
\makeatother
\makeatletter
\@ifpackageloaded{caption}{}{\usepackage{caption}}
\@ifpackageloaded{subcaption}{}{\usepackage{subcaption}}
\makeatother

\ifLuaTeX
  \usepackage{selnolig}  % disable illegal ligatures
\fi
\usepackage{bookmark}

\IfFileExists{xurl.sty}{\usepackage{xurl}}{} % add URL line breaks if available
\urlstyle{same} % disable monospaced font for URLs
\hypersetup{
  pdftitle={This is the title},
  colorlinks=true,
  linkcolor={blue},
  filecolor={Maroon},
  citecolor={Blue},
  urlcolor={Blue},
  pdfcreator={LaTeX via pandoc}}


\title{This is the title}
\author{}
\date{}

\begin{document}
\maketitle
\begin{abstract}
Heat maps are a common data visualization display when evaluating a
continuous response across two predictor variables. Typical
presentations of heat maps are in two-dimensional formats, relying on
color to convey the response. With advances in computer graphics and
physical renderings, we are able to convert heat maps into three
dimensional representations, using height to represent the response
variable. However, the dimensionality of heat maps for the use of
conveying statistical information has not been widely studied. In this
paper, we outline an experiment for comparing the digital and physical
dimensionality of heat maps using 2D and 3D representations, using
3D-printed heat maps to evaluate
\end{abstract}


\textbf{Keywords:} graphics, heat maps, 3D-printed charts

\section{Introduction}\label{introduction}

In the 20th century, advancements in technology made data visualizations
increasingly more affordable and accessible to a broader population. The
primary change in formal data visualizations was from hand-crafted
charts to computer-rendered charts (Tukey 1965), yet other technological
advances have allowed for data visualizations to enter other mediums.
These include the ability to effectively use the 3-dimensional (3D)
world around us with the novel use of 3D-printed charts. This newer type
of visualization has gained a small amount of traction in recent years
as a method of producing tangible charts.

Current literature is mainly limited to 2D projections of 3D charts,
comparisons of 2D and 3D charts show mixed results. In 2D projections,
this is mainly attributed to the purpose of the depth cues in 3D charts.
When the increased dimensionality does not include additional
information, information extraction tends to be worse for 3D charts than
2D charts (Zacks et al. 1998). However, the inclusion of information in
the depth axis tends to lower error rates of numerical estimations and
increase memorability of results (Barfield and Robless 1989; Jeong et
al. 2025). These discrepancies suggest that 3D charts have situational
use cases where they can outperform the delivery of statistical
information of other types of charts.

When depth is used for conveying information, we hypothesize that the
benefits observed from 2D displays of 3D charts will increase for charts
rendered in our 3D world. Our study partially replicates the proceedure
of well-known statistical studies in data visualizations and is designed
to provide some of the initial insights on 3D-printing for visualizing
data.

\section{Methods}\label{methods}

Our study is designed to evaluate and expand the literature on numerical
estimation of ratios for 3D charts. To accommodate discrepancies with
perceived magnitudes of heights and volumes (Stevens and Stevens 1986;
Cleveland and McGill 1984), we use the method of constant stimuli for
selecting values to be used for ratio estimations. All stimuli pairs are
placed onto a single heat maps to reduce the number of 3D-printed charts
required.

\subsection{Selection of Stimuli
Values}\label{selection-of-stimuli-values}

Stimuli values are chosen so that the constant stimuli is at 50 units.
We set the maximum stimuli at 90 units and equally partition the ratios
between \(50/90\approx 0.56\) and \(50/50=1\) to get stimuli values
between 50 units and 90 units, giving ratios of 0.56, 0.67, 0.78, and
0.89. We then set 50 units as the maximum value and use the same ratios
to obtain stimuli values that are less than 50 units. Additionally, we
include a stimuli pair where both values are the same, resulting in 9
total pairs of stimuli.

The heat map was controlled to be a square grid across \(X\) and \(Y\)
coordinates, where \(X=1,2,\dots,10\) and \(Y=1,2,\dots 10\). Randomized
values are generated using a mixture distribution of a uniform
distribution and the formula of a half-sphere centered on the heat map
coordinate grid, given by:

\[
Z=c\cdot U(0, 100)+(1-c)\cdot g(f_i(X,Y))
\]

where

\begin{itemize}
\tightlist
\item
  \(g(Z)=100\cdot\frac{Z-\min(Z)}{\max(Z)-\min(Z)}\)
\item
  \(f_1(X,Y)=\sqrt{7^2-(X-\bar{X})-(Y-\bar{Y})}\)
\item
  \(f_2(X,Y)=\sqrt{7^2-(X-\bar{X})+(Y-\bar{Y})}\)
\item
  \(c=0.3\)
\end{itemize}

Placement of the controlled stimuli values was performed via simulation
to try and make their placement look as natural as possible. The
non-constant stimuli is initially placed on the coordinate that
minimizes the difference between the randomly generated value and the
stimuli. Then the constant stimuli is placed on a coordinate within 3 or
4 grid units away from the non-constant stimuli that minimizes the
difference between the randomized value and the constant stimuli. Twenty
heat maps were generated for each \(f_i\) and Chi-squared tests were
used to select the heat map where stimuli values were reasonably
distributed across the coordinate grid.

\subsection{Chart types}\label{chart-types}

We considered three types of charts in our experiment: 2D-digital (2dd),
3D-digital (3dd), and 3D-printed (3dp). We used the 3dp chart as a
baseline for the construction of 2dd and 3dd charts. The 3dp chart was
designed so that each tile is 1 centimeter\textsuperscript{2} with a
maximum height of 10-centimeters. We printed these charts with either
solid or gradient filaments. The print files for 3dp charts were
converted into WebGL displays (Murdoch and Adler 2023), creating exact
one-to-one digital displays of the 3dp charts. Without a direct mapping
of color to height, we designed the 2dd chart to have a gradient fill
that mirrored the lighting conditions of our initial 3dp charts,
interpolated with the \texttt{scale\_fill\_gradient()} function from the
ggplot2 package (Wickham 2016) using \#0C2841 and \#66D9FF as the color
range.

\begin{figure}

\begin{minipage}{0.33\linewidth}

\centering{

\includegraphics{images/examp2dd.png}

}

\subcaption{\label{fig-2dd}2dd}

\end{minipage}%
%
\begin{minipage}{0.33\linewidth}

\centering{

\includegraphics{images/examp3dd.png}

}

\subcaption{\label{fig-2dd}3dd}

\end{minipage}%
%
\begin{minipage}{0.33\linewidth}

\centering{

\includegraphics{images/examp3dp.png}

}

\subcaption{\label{fig-2dd}3dp}

\end{minipage}%

\caption{\label{fig-charts}Chart types representing data set 1.}

\end{figure}%

\subsection{Experimental Design}\label{experimental-design}

With a completely randomized design, participants would complete 54
trials (3 chart types × 9 stimuli pairs × 2 data sets). Instead, we
opted for a balanced incomplete block design to hopefully reduce the
effects contributed by fatigue, using 4 of the 9 stimuli pairs for each
participant to construct the blocks. This results in 24 trials per
participant, which hopefully improved the overall quality of the
responses. Trials were presented in a randomized order of chart type ×
data set, where all four stimuli pairs were presented in a randomized
order before moving onto the next chart type × data set.

Following the seminal work of Cleveland and McGill (1984) in data
visualization practices, we asked participants two modified questions
for each trial. The first question asked participants which value in the
identified stimuli pair is larger. This question served as an attention
check to determine if the participant was correctly following the
instructions for their ratio estimation. The second question asked
participants to use a slider to estimate the quantity of the smaller
value if the larger value in the stimuli pair represented 100 units.
Ratio estimations take the form of \(A/B\), where \(A\leq B\).

\section{Discussion}\label{discussion}

In this paper, we described an experiment that explores the relationship
of the physical dimensionality in renderings of heat maps. While 2D
displays of 2D and 3D data visualizations have been widely explored, our
work contributes to the broader understanding of physical dimensionality
of 2D and 3D charts. At the conclusion of the study, we will release a
repository to encourage replication and sharing of experimental results.

\phantomsection\label{refs}
\begin{CSLReferences}{1}{0}
\bibitem[\citeproctext]{ref-barfieldEffectsTwoThreedimensional1989}
Barfield, Woodrow, and Robert Robless. 1989. {``The Effects of Two- or
Three-Dimensional Graphics on the Problem-Solving Performance of
Experienced and Novice Decision Makers.''} \emph{Behaviour \&
Information Technology} 8 (5): 369--85.
\url{https://doi.org/10.1080/01449298908914567}.

\bibitem[\citeproctext]{ref-cleveland1984}
Cleveland, William S., and Robert McGill. 1984. {``Graphical Perception:
Theory, Experimentation, and Application to the Development of Graphical
Methods.''} \emph{Journal of the American Statistical Association} 79
(387): 531--54. \url{https://doi.org/10.1080/01621459.1984.10478080}.

\bibitem[\citeproctext]{ref-jeongComparativeStudy2D2025}
Jeong, Jongwook, Ayoung Choi, Dongwoo Kang, and Youn Kyu Lee. 2025. {``A
Comparative Study of {2D} Vs. {3D} Chart Visualizations in Virtual
Reality.''} \emph{Journal of Visualization} 28 (1): 239--53.
\url{https://doi.org/10.1007/s12650-024-01033-6}.

\bibitem[\citeproctext]{ref-rgl}
Murdoch, Duncan, and Daniel Adler. 2023. \emph{Rgl: 3D Visualization
Using OpenGL}.

\bibitem[\citeproctext]{ref-stevens1986}
Stevens, S. S., and Geraldine Stevens. 1986. \emph{Psychophysics:
Introduction to Its Perceptual, Neural, and Social Prospects}. New
Brunswick, U.S.A: Transaction Books.

\bibitem[\citeproctext]{ref-tukey1965}
Tukey, John W. 1965. {``The Technical Tools of Statistics.''} \emph{The
American Statistician} 19 (2): 23--28.
\url{https://doi.org/10.2307/2682374}.

\bibitem[\citeproctext]{ref-ggplot2}
Wickham, Hadley. 2016. {``Ggplot2: Elegant Graphics for Data
Analysis.''} \url{https://ggplot2.tidyverse.org}.

\bibitem[\citeproctext]{ref-zacks1998}
Zacks, Jeff, Ellen Levy, Barbara Tversky, and Diane J. Schiano. 1998.
{``Reading Bar Graphs: Effects of Extraneous Depth Cues and Graphical
Context.''} \emph{Journal of Experimental Psychology: Applied} 4 (2):
119--38. \url{https://doi.org/10.1037/1076-898X.4.2.119}.

\end{CSLReferences}




\end{document}
